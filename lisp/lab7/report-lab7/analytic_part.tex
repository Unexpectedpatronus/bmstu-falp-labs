\chapter{Аналитическая часть}

Функции Фибоначчи и факториала имеют широкое применение в различных областях, таких как математика, программирование и алгоритмы. Рекурсивные алгоритмы для их вычисления позволяют наглядно продемонстрировать применение рекурсии и её особенности.

\subsection{Функция Фибоначчи}
Числа Фибоначчи \( F(n) \) определяются рекурсивной формулой:
\begin{equation}
	F(n) = F(n-1) + F(n-2),
\end{equation}
где \( F(0) = 0 \) и \( F(1) = 1 \).

\subsection{Функция факториала}
Факториал числа \( n \), обозначаемый как \( n! \), определяется произведением всех целых чисел от 1 до \( n \):
\begin{equation}
	n! = n \times (n - 1) \times \dots \times 1,
\end{equation}
где \( 0! = 1 \).

Обычная рекурсия для этих функций проста, но менее оптимальна, так как при больших значениях \( n \) может привести к переполнению стека вызовов. Хвостовая рекурсия более эффективна за счёт того, что сохраняет результат в аккумуляторе, что позволяет компилятору оптимизировать выполнение.


\clearpage
