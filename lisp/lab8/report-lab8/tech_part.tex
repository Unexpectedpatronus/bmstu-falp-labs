\chapter{Технологическая часть}

\section{Описание функций}
Программа реализована с использованием функций на языке Common Lisp. Основные функции:
\begin{itemize}
	\item \texttt{discriminant} — вычисляет дискриминант уравнения.
	\item \texttt{quadratic-solution} — находит корни уравнения в зависимости от значения дискриминанта.
	\item \texttt{solve-quadratic-apply} и \texttt{solve-quadratic-funcall} — вызывают функцию решения уравнения с использованием функций \texttt{apply} и \texttt{funcall}.
\end{itemize}

\section{Код программы}
\begin{lstlisting}[language=Lisp, caption={Функции для нахождения корней квадратного уравнения}]
	(defun discriminant (a b c)
	  (- (* b b) (* 4 a c)))
	
	(defun quadratic-solution (a b d)
	  (let ((sqrt-d (sqrt (abs d)))
	        (denom (* 2 a)))
	    (if (>= d 0)
	        (list (/ (+ (- b) sqrt-d) denom)
	              (/ (- (- b) sqrt-d) denom))
	        (list (complex (/ (- b) denom) (/ sqrt-d denom))
          	      (complex (/ (- b) denom) (- (/ sqrt-d denom)))))))
	
	(defun solve-quadratic-apply (a b c)
	  (let ((d (discriminant a b c)))
	    (apply #'quadratic-solution (list a b d))))
	
	(defun solve-quadratic-funcall (a b c)
	  (let ((d (discriminant a b c)))
	    (funcall #'quadratic-solution a b d)))
\end{lstlisting}


