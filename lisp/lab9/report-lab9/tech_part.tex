\chapter{Технологическая часть}

В данном разделе приведены листинги программного кода, реализующего задачи, рассмотренные в аналитическом разделе, а также примеры использования функций с выводом полученных результатов.

\subsection{Листинг функций}
\begin{lstlisting}[language=Lisp]
	(defun cross-product (vec1 vec2)
	  (let* ((indices '((1 2) (2 0) (0 1)))
	    (components (mapcar (lambda (idx)
	      (- (* (nth (first idx) vec1) (nth (second idx) vec2))
	         (* (nth (second idx) vec1) (nth (first idx) vec2))))
	               indices)))
	  components))
	
	(defun cartesian-product (list1 list2)
	  (reduce #'append
	      (mapcar (lambda (x) (mapcar (lambda (y) (list x y)) list2))
      	       list1)))
	
	(defun base-n-to-decimal (digits base)
	  (reduce (lambda (acc digit) 
	      (+ (* acc base) digit)) digits))
	
\end{lstlisting}

\subsection{Примеры использования и результаты тестов}
\subsubsection*{Пример 1: Векторное произведение}
Рассмотрим результат выполнения функции для вычисления векторного произведения двух векторов:

\begin{lstlisting}[language=Lisp,caption={Векторное произведение}]
	(cross-product '(1 2 3) '(4 5 6)) 
\end{lstlisting}

Результат: \((-3 \; 6 \; -3)\).

\subsubsection*{Пример 2: Декартово произведение множеств}
Рассмотрим результат выполнения функции для вычисления декартова произведения множеств \((1, 2)\) и \((3, 4)\):

\begin{lstlisting}[language=Lisp,caption={Декартово произведение}]
	(cartesian-product '(1 2) '(3 4)) 
\end{lstlisting}

Результат: \(((1 \; 3) \; (1 \; 4) \; (2 \; 3) \; (2 \; 4))\).

\subsubsection*{Пример 3: Перевод числа из \(N\)-ричной системы счисления в десятичную}
Рассмотрим перевод числа \(1011\) из двоичной системы в десятичную:

\begin{lstlisting}[language=Lisp,caption={Перевод числа в десятичную систему счисления}]
	(base-n-to-decimal '(1 0 1 1) 2) 
\end{lstlisting}

Результат: \(11\).

