\chapter{Аналитическая часть}

В данном разделе рассматриваются теоретические сведения по задачам, для которых требуется разработать программные реализации.

\subsection{Векторное произведение векторов}
Векторное произведение двух векторов \(\mathbf{A}\) и \(\mathbf{B}\) в трёхмерном пространстве является вектором, перпендикулярным к обоим исходным векторам. Пусть \(\mathbf{A} = (a_1, a_2, a_3)\) и \(\mathbf{B} = (b_1, b_2, b_3)\), тогда результат векторного произведения можно выразить как новый вектор \(\mathbf{C}\), координаты которого вычисляются по следующим формулам:
\[
c_1 = a_2 \cdot b_3 - a_3 \cdot b_2,
\]
\[
c_2 = a_3 \cdot b_1 - a_1 \cdot b_3,
\]
\[
c_3 = a_1 \cdot b_2 - a_2 \cdot b_1.
\]
Векторное произведение имеет широкое применение в физике, инженерии и компьютерной графике, например, для вычисления нормалей к поверхностям.

\subsection{Декартово произведение множеств}
Декартово произведение двух множеств \(A\) и \(B\) — это множество всех возможных упорядоченных пар \((a, b)\), где \(a \in A\) и \(b \in B\). Декартово произведение обозначается как \(A \times B\) и определяется по формуле:
\[
A \times B = \{ (a, b) \; | \; a \in A, \; b \in B \}.
\]
Декартово произведение полезно в различных задачах теории множеств, реляционных баз данных и комбинаторных вычислений.

\subsection{Перевод числа из \(N\)-ричной системы счисления в десятичную}
Для перевода числа из \(N\)-ричной системы счисления в десятичную применяется следующая формула. Пусть число в \(N\)-ричной системе представлено в виде набора цифр \( (d_0, d_1, \ldots, d_{k-1}) \), тогда его значение в десятичной системе определяется как:
\[
\text{decimal} = d_0 \cdot N^{k-1} + d_1 \cdot N^{k-2} + \ldots + d_{k-1} \cdot N^0.
\]
Этот метод позволяет представлять числа в различных системах счисления и переводить их в стандартную десятичную систему.